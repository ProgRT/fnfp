\def\fauxneg{"data/Faux_négatifs/1555833401655-parsed.dat"}
\def\fauxnegbis{"data/Faux_négatifs/1555833504659-parsed.dat"}
\def\fauxnegtri{"data/Faux_négatifs/1558262528660-parsed.dat"}
\def\fauxnegcorige{"data/Faux_négatifs/1558262056661-parsed.dat"}

\tikzset{
	annot/.style={
		red,
		<-,
		draw,
		thick,
		circle,
		minimum width=30,
		inner sep=0,
		yshift=-2,
		font=\normalfont
	}
}

\begin{frame}{Qu'est-ce qui cloche ?}
	\begin{tikzpicture}[]

		\begin{groupplot}[
				group style={
					group name=my group,
					group size=1 by 3,
				},
				xmax=16,
				every axis style/.append style={
					minor x ticks num=3,
					grid=both,
				}
				height=0.5\textheight,
				xlabel=Temps (s),
				anchor=center,
				at={(0,0)}
			]

			\nextgroupplot[
				ymin=0,
				ylabel=Pression (\cmh),
				height=0.32\textheight
			]

			\addplot [] table[x index=0, y index=2] {\fauxnegtri};

			\nextgroupplot[ylabel=Débit (l/m), height=0.45\textheight]
				\addplot [] table[x index=0, y index=3] {\fauxnegtri};
				\coordinate (myanchor) at(12.75,0) ;
				\node<2,3>[annot] at (4,0) {};
				\node<2,3>[annot] at (11.8,0) {};

			\only<3->{
				\nextgroupplot[ylabel=EADI (mcv), height=0.31\textheight]
					\addplot[] table[x index=0, y index=5] {\fauxnegtri};
					\coordinate (x1) at(axis cs:3.05,0);
					\coordinate (x2) at(axis cs:8,0);
					\coordinate (x3) at(axis cs:8.75,0);
			}

		\end{groupplot}

		\only<4-5>{
			\draw [red, dashed] (x1) -- ++(0, \textheight-66pt);
		}
		
		\only<5>{
			\draw [red, dashed] (x2) -- ++(0, \textheight-66pt) coordinate(c1);
			\draw [red, dashed] (x3) -- ++(0, \textheight-66pt) coordinate(c2);
			\draw [<->, red] (c1) -- (c2) node[midway, above] {0,75 s};
		}

		\node <6>[
			postit,
		] at (myanchor) {
			\begin{tabular}{l @{\ :\ \ } r @{ } >{\footnotesize}l }
				Volume courant   & 450 & ml \\
				PEP              & 5   & \cmh \\
				F resp.          & 12  & resp/min \\
				Conc. \ox        & 35  & \% \\
				Pente tps insp.  & 0   & s \\
				Durée de plateau & 0   & s \\
				Décl. en débit   & 0   & l/min \\
				Ti               & 0,5 & s
			\end{tabular}
		};
	\end{tikzpicture}
\end{frame}
